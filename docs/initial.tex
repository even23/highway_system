\documentclass[a4paper]{article}

%% Language and font encodings
\usepackage[utf8]{inputenc}
\usepackage[T1]{fontenc}
\usepackage{polski}
\usepackage{amssymb}


%% Sets page size and margins
\usepackage[a4paper,top=3cm,bottom=3cm,left=3.5cm,right=3.5cm,marginparwidth=2cm]{geometry}

%% Useful packages
\usepackage{amsmath}
\usepackage{graphicx}
\usepackage[colorinlistoftodos]{todonotes}
\usepackage[colorlinks=true, allcolors=blue]{hyperref}

\title{ALHE - Specyfikacja projektu}
\author{Kaczmarek Kamil, Lewczuk Grzegorz}

\begin{document}
\maketitle

\begin{abstract}
Analiza problemu utworzenia sieci autostrad pomiędzy miastami oraz implementacja algorytmu optymalizującego rozwiązanie
\end{abstract}

\section{Opis zagadnienia}

\begin{quote}
Przygotować algorytm poszukujący optymalnej sieci autostrad tworzącą siatkę połączeń pomiędzy miastami z danego zbioru - położenia miast są nam znane. Rozwiązanie powinno uwzględniać miejsca zjazdów (nie mogą one znajdować się zbyt blisko siebie) oraz pozwalać na przecinanie się autostrad nie tylko w miastach.
\end{quote}

\subsection{Założenia}
\begin{quote}
\begin{enumerate}
\item Przestrzeń - jest to przestrzeń $\mathbb{Z}^2$
\item Miasto - punkt [\textit{x, y}] znajdujący się w przestrzeni
\item Parametr M określający liczbę miast (miasta są znane)
\item Punkt zjazdu - punkt [\textit{x, y}] znajdujęcy się na autostradzie 
\item Zjazd - odcinek w linii prostej łączący miasto z najbliższym Punktem zjazdu na autostradzie
\item Parametr K określający liczbę punktów składających się na odcinki sieci autostrad
\item Parametr D określający minimalną odległość pomiędzy dwoma Punktami zjazdu (jeśli wynaczone punkty zjazdu znajdą się zbyt blisko siebie to zostaną zastąpione jednym)
\end{enumerate}
\end{quote}
\section{Przestrzeń poszukiwań i sąsiedztwo}

\begin{quote}
Element przestrzeni poszukiwań to wektor K-elementowy, na który skłądają się krotki postaci [\textit{x, y, w}] reprezentujące wierzchołki łamanej tworzącej autostradę. \linebreak Para (\textit{x, y}) stanowi współrzędne tego punktu na płaszczyźnie. Zmienna \textit{w} określa, który wierzchołek jest połączony odcinkiem z aktualnym. Przyjmuje ona wartości od \textit{-1} do \textit{K-1} i oznacza, do wierzchołka o jakim indeksie występuje połączenie. Wierzchołki numerowane są od 0. (np. \textit{w}=1 oznacza, że rozpatrywany wierzchołek łączy się z wierzchołkiem o indeksie 1, \textit{w}=-1 oznacza, że wierzchołek nie łączy się z żadnym innym). Tak zdefiniowana przestrzeń dopuszcza obustronne połączenia, ale zostaną one wyeliminowane przez metaheurystykę - nie wnoszą nic do struktury sieci autostad, a jedynie podnoszą koszt. Przestrzeń oprócz rozwiązań prawidłowych zawiera również rozwiązania niedopuszczalne, czyli takie, w których nie istnieje połącznie między wszystkimi miastami. Warunek ten zostanie osiągnięty dzięki odpowiednio sformułowanej funkcji celu. Sasiądami \textit{k}-tego stopnia będziemy nazywać wektory różniące się \textit{k} krotkami.
\end{quote}

\section{Funkcja celu}

\begin{quote}
Parametry funkcji celu:
\begin{enumerate}
\item Położenie miast
\item Wektor określający sieć autostrad
\item Funkcja kosztu budowy autostrady, zależna od długości
\item Funkcja kosztu budowy zjazdu, zależna od długości 
\end{enumerate}

Wartość funkcji celu będzie liczona na podstawie:
\begin{enumerate}
\item Kosztu budowy sieci autostrad
\item Kosztu budowy zjazdów
\item Kary związanej z niezapewnieniem spójności autostrady
\end{enumerate}f

Funkcja kosztu budowy autostrady ${f(x)}$ będzie rosła liniowo, zależnie od ${k}$ kosztu jednego kilometra autostrady\newline
Funkcja kosztu budowy zjazdu ${g(x)}$ musi szybciej rosnąć od ${f(x)}$ (wielomianowa lub wykładnicza), jednak do pewnej wartości ${x}$ musi być bardziej opłacalna niż ${f(x)}$, aby zapewnić jak najkrótsze zjazdy z autostrad\newline
Dodatkowo sprawdzane będzie połączenie pomiędzy każdymi dwoma dowolnymi miastami. Jeżeli takie połączenie nie będzie dostępne, do funkcj celu zostanie dodana kara ${P}$ dostatecznie duża, aby to rozwiązanie odrzucić\newline
Wartość funkcji celu będzie sumą tych czynników\newline
Optymalizacja będzie polegała na minimalizacji kosztu budowy sieci autostrad

\subsection{Przykład}
Niech ${V}$ będzie wektorem Miast o długości ${M}$\newline
Niech ${A}$ będzie wektorem krotek [\textit{x, y, w}] o długości ${K}$\newline
Niech ${f(a)}$ będzie funkcją kosztu budowy odcinka autostrady o długości \textit{a}\newline
Niech ${g(b)}$ będzie funkcją kosztu budowy zjazdu o długości \textit{b}\newline
Niech ${P}$ będzie karą braku połączenia pomiędzy miastami\newline
Niech ${k(V, A)}$ będzie funkcją kosztu budowy sieci autostrad\newline

Dla:\newline
${M = 3}$\newline
${V = [(1, 1), (3, 0), (2, 4)]}$\newline
${K = 3}$\newline
${A = [(1, 2, 1), (2, 2, 2), (3, 2, -1)]}$\newline
${f(a) = a}$\newline
${g(b) = 2^b - 1}$\newline
${P = 10000}$\newline

\begin{tikzpicture}[x=1cm,y=1cm]

\draw[latex-latex, thin, draw=gray] (0,0)--(4,0) node [right] {$x$};
\draw[latex-latex, thin, draw=gray] (0,0)--(0,4) node [above] {$y$};
\draw[thick] (1, 2)--(3, 2);
\node [blue] at (1, 2) {$\circ$};
\node [blue] at (2, 2) {$\circ$};
\node [blue] at (3, 2) {$\circ$};

\node [blue] at (1, 1) {\textbullet};
\draw [dashed, blue] (1, 1)--(1, 2);
\node [blue] at (3, 0) {\textbullet};
\draw [dashed, blue] (3, 0)--(3, 2);
\node [blue] at (2, 4) {\textbullet};
\draw [dashed, blue] (2, 4)--(2, 2);
\draw [dotted, gray] (0, 0) grid (4,4);

\end{tikzpicture}

Długość autostrady: ${7}$. Długość zjazdów ${1, 2, 2}$\newline
${f(a) = 7}$\newline
${g(z_1) = 2^1 - 1 = 1}$\newline
${g(z_2) = g(z_3) = 2^2 - 1 = 3}$\newline
${g(b) = 1 + 3 + 3 = 7}$\newline
${k(V, A) = 7 + 7 = 14}$\newline

Spójność autostrady jest zapewniona, więc nie została doliczona kara ${P}$.
\end{quote}

\section{Metody optymalizacji}

\begin{quote}
Do rozwiązania zadania planowane jest użycie metody przeszukiwania ze zmiennym sąsiedztwem
(VNS) oraz symulowanego wyżarzania.
\end{quote}

\section{Sposób przeprowadzania eksperymetów}

\begin{enumerate}
\item Testy w początkowej fazie eksperymentów będą przeprowadzana na małych problemach. Dzięki temu można będzie uzyskiwane podczas nich wyniki porównać z tymi uzyskanymi ręcznie i upewnić się, że algorytm działa zgodnie z oczekiwaniami i daje poprawne rezultaty.
\item Testy właściwe zostaną przeprowadzone dla parametru M równego 10, 20, 50.
\item Wyniki eksperymentów będą przedstawiane w formie tekstowej oraz graficznej, aby w łatwy sposób zweryfikować wyniki.
\end{enumerate}

\end{document}